\documentclass[english]{article}

\usepackage{graphicx}
\usepackage{grffile}
\usepackage[T1]{fontenc}
\usepackage{babel}
\usepackage{wrapfig}
\usepackage{hyperref}

\date{\today}

\graphicspath{{Pictures/}}
\begin{document}	
	\begin{titlepage}
		\pagenumbering{gobble}
		\begin{figure}[!t]
			\includegraphics[width=\linewidth]{up_logo.png}
		\end{figure}
		\vspace*{\stretch{1.0}}
		\begin{center}
			\huge{Project: Harvest}\\
			\large{Client: Subtrop}\\
			\vspace{10mm}
			\huge{Team: A-Cube-N}\\
		\end{center}
		\begin{center}
			\begin{tabular}{ c c c }
				Dunkley, Nathan & Grobler, Arno & Lochner, Amy \\
				\texttt{14145759} & \texttt{STUDENT NR} & \texttt{14038600}\\
				& Maree, Armand &\\
				& \texttt{12017800} &
			\end{tabular}
		\end{center}
		\begin{center}
			Department of Computer Science, University of Pretoria
		\end{center}
		\begin{center}
			\today
			\begin{figure}[!h]
				\includegraphics[width=\linewidth]{team.jpg}
			\end{figure}
		\end{center}
		\vspace*{\stretch{2.0}}
	\end{titlepage}
	\newpage
	\tableofcontents
	\newpage
	\pagenumbering{arabic}
	\section{The Team}
		\subsection{Nathan Dunkley}
		
		\subsection{Arno Gerber}
		
		\subsection{Amy Lochner}
		\begin{wrapfigure}{l}{5cm}
			\begin{center}
				\includegraphics[width=8cm, height=4.5cm, angle=90]{amy.jpg}
			\end{center}
		\end{wrapfigure}
		\paragraph{Interests and Hobbies}
		My interests include music, classic cars, cooking, traveling, breeding Shetland sheepdogs. My hobbies include reading, playing piano, camping, 4x4ing, tennis, training my dog, mountain biking and horse riding.
		
		\paragraph{Technical Skills}
		I am good at determining functional requirements of a system. I can place myself in the users shoes, this is valuable when determining how the user will intend to use a system. I can follow business logic easily and I have experience in databasing, Informatics, Statistics, Mathematics, multiple programming languages and Human Computer Interaction.
		
		\paragraph{Past Experience}
		I have built a fully functional, responsive website. I have helped a company modify their website. I have also observed (by job shadowing) the process of creating a system for a business and have noticed which qualities have caused them to excel and which have caused them to fail. I intend to use that knowledge to keep our team constantly progressing forward.
		
		\paragraph{Non-Technical Strengths}
		\begin{itemize}
			\setlength\itemsep{0.2em}
			\item Organized
			\item Good at prioritising 
			\item Team player
			\item Good leader
			\item Optimistic
			\item Quick learner
			\item Determined
		\end{itemize}
		
		\paragraph{Motivation}
		To do..
		
		\subsection{Armand Maree}
			\begin{wrapfigure}{l}{5.1cm}
				\begin{center}
					\includegraphics[width=5cm]{armand.jpg}
				\end{center}
			\end{wrapfigure}
			\paragraph{Interests and Hobbies}
			During my off time I like to socialize with friends and enjoy watching sports. I also like solving puzzles to keep my brain active during holidays.\\
			Tutoring scholars and university students has become a passion for me. I always look forward to these sessions.
			
			\paragraph{Technical Skills}
			I am good at solving complex problems and building data structures. I believe this is a valuable skill to complete any project, especially in the field of computer science.
			
			\paragraph{Past Experience}
			I have developed websites for other start up companies and I also have a website of my own (\href{http://www.codehaven.co.za}{www.codehaven.co.za}). I also have some Android developing experience I gained from side projects.
			
			\paragraph{Non-Technical Strengths}
			\begin{itemize}
				\setlength\itemsep{0.2em}
				\item Good leader
				\item Fast learner
				\item Team player
				\item Good communicator
				\item Passionate
				\item Problem solver
			\end{itemize}
			
			\paragraph{Motivation}
			As soon as I read the project specifictions I was very eager to apply for it. The most attractive part of the project was the fact that is so applicable to the industry. Developing a system that would directly help the country, and more specifically farmers, sounded like a great idea. Since I am also a huge advocate of open source software, the proposal by Subtrop to make this software available for other farmers was a big attraction.
			
	\newpage
	\section{Project Execution}
		\subsection{Development Methodology}
			\paragraph\indent
			We are plainng on using the iterative development methodology as this would allow us to more frequently display working progress to the system to the client compared to the Waterfall methodology. This methodology will also allow us to have larger, but still manageable, portions of the work done between each meeting. We believe this is essential in order to make faster progress while still being able to make changes to the system should the requirements change. See figure \ref{fig:developmentMethodologies}.
			
			\begin{figure}[!h]
				\includegraphics[width=\linewidth]{developmentMethodologies.png}
				\caption{Waterfall vs Iterative vs Extreme Programming methodologies.}
				\label{fig:developmentMethodologies}
			\end{figure}
		
		\subsection{Client Updates}
			Since Subtrop is located in Tzaneen it would be more practical to perhaps have Skype/Team Viewer meetings to do demos of the current progress of the system. Less frequent face-to-face meeting could be arranged in order for Subtrop and developers (students) to discuss important milestones in the project, should it be necessary.
			
		\subsection{Initial Ideas}
		
		\subsection{Potential Technologies}
		
		\subsection{Deliverables}
			A-Cube-N will provide all the source code in order to join Subtrob in making this software available for other subtropical farmers by means of open source. We will also provide detailed documentation on the system to facilitate future developments. And finally we will provide a user manual that will provide instructions on how the system is used and how it is set up.
\end{document}
