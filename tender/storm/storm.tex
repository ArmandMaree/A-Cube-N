\documentclass[english]{article}

\usepackage{graphicx}
\usepackage{grffile}
\usepackage[T1]{fontenc}
\usepackage{babel}
\usepackage{wrapfig}
\usepackage{hyperref}

\date{\today}

\graphicspath{{Pictures/}}
\begin{document}	
	\begin{titlepage}
		\pagenumbering{gobble}
		\begin{figure}[!t]
			\includegraphics[width=\linewidth]{up_logo.png}
		\end{figure}
		\vspace*{\stretch{1.0}}
		\begin{center}
			\huge{Project: STORM}\\
			\large{Client: Ms Linda Marshall}\\
			\vspace{10mm}
			\huge{Team: A-Cube-N}\\
		\end{center}
		\begin{center}
			\begin{tabular}{ c c c }
				Dunkley, Nathan & Grobler, Arno & Lochner, Amy \\
				\texttt{14145759} & \texttt{14011396} & \texttt{14038600}\\
				& Maree, Armand &\\
				& \texttt{12017800} &
			\end{tabular}
		\end{center}
		\begin{center}
			Department of Computer Science, University of Pretoria
		\end{center}
		\begin{center}
			\today
			\begin{figure}[!h]
				\includegraphics[width=\linewidth]{team.jpg}
			\end{figure}
		\end{center}
		\vspace*{\stretch{2.0}}
	\end{titlepage}
	\newpage
	\tableofcontents
	\newpage
	\pagenumbering{arabic}
	\section{The Team}
		\subsection{Nathan Dunkley}
			\begin{wrapfigure}{l}{5cm}
				\begin{center}
					\includegraphics[width=8cm, height=4.5cm, angle=90]{nathan.jpg}
				\end{center}
			\end{wrapfigure}
			\paragraph{Interests and Hobbies}
			My interests include playing and watching sport, specifically motorsport (Formula One, World Endurance Championship), cricket, tennis and golf. I play tennis twice a week at a club. I also like to listen to music and read books as well as play games on PC.
	
			\paragraph{Technical Skills}
			I'm more of a follower than a leader and I'm good at getting on with work once the tasks have been delegated to the members of the group. I enjoy working on tasks that interest me and don't mind working long hours to get it done, once I've put my mind to it. I have some experience in multiple programming languages and I enjoy learning new skills when I can. I also enjoy solving problems.
	
			\paragraph{Past Experience}
			Minor experience in Android Development.
	
			\paragraph{Non-Technical Strengths}
			\begin{itemize}
				\setlength\itemsep{0.2em}
				\item Fast Learner
				\item Willing to Learn
				\item Flexible
			\end{itemize}
	
			\paragraph{Motivation}
			To do..
		
		\newpage
		\subsection{Arno Grobler}
			\begin{wrapfigure}{l}{5.1cm}
				\begin{center}
					\includegraphics[width=5cm]{arno.jpg}
				\end{center}
			\end{wrapfigure}
			\paragraph{Interests and Hobbies}
			My interests include collecting music, long distance running, painting and drawing, reading, computer games and obviously spending most of my days programming. Not only do I want to program as a profession, it is also a hobby for me. Integrating my other hobbies into my programming is my passion.
			
			\paragraph{Technical Skills}
            I pride myself in always looking for new skills and for me, learning a new technical skill is the best part of the experience. I enjoy making my projects look visually pleasing and spend as much time making a working, functional program as I do making it look good. I have good logical and problem solving skills and enjoy problems presented to me in computer science. My technical skills stem from Mathematics and computer science, especially those skills from data structures and algorithms and programming logic.
			
			\paragraph{Past Experience}
            I have created static websites for companies before, my most recent one is (\href{http://bodytalkbethlehem.com/}{http://bodytalkbethlehem.com/}) and (\href{http://honeydewpools.co.nf/}{http://honeydewpools.co.nf/}).
			
			\paragraph{Non-Technical Strengths}
			\begin{itemize}
				\setlength\itemsep{0.2em}
			        \item Eager learner
			        \item Organised 
			        \item Good time management
			        \item Good communication skills
			        \item Creative
			\end{itemize}
			
			\paragraph{Motivation}
			To do..
		
		\newpage
		\subsection{Amy Lochner}
		    \begin{wrapfigure}{l}{5cm}
				\begin{center}
					\includegraphics[width=8cm, height=4.5cm, angle=90]{amy.jpg}
				\end{center}
			\end{wrapfigure}
			\paragraph{Interests and Hobbies}
			My interests include music, classic cars, cooking, traveling, breeding Shetland sheepdogs. My hobbies include reading, playing piano, camping, 4x4ing, tennis, training my dog, mountain biking and horse riding.
			
			\paragraph{Technical Skills}
			I am good at determining functional requirements of a system. I can place myself in the users shoes, this is valuable when determining how the user will intend to use a system. I can follow business logic easily and I have experience in databasing, Informatics, Statistics, Mathematics, multiple programming languages and Human Computer Interaction.
			
			\paragraph{Past Experience}
			I have built a fully functional, responsive website. I have helped a company modify their website. I have also observed (by job shadowing) the process of creating a system for a business and have noticed which qualities have caused them to excel and which have caused them to fail. I intend to use that knowledge to keep our team constantly progressing forward.
			
			\paragraph{Non-Technical Strengths}
			\begin{itemize}
				\setlength\itemsep{0.2em}
			        \item Organized
			        \item Good at prioritising 
			        \item Team player
			        \item Good leader
			        \item Optimistic
			        \item Quick learner
			        \item Determined
			\end{itemize}
			
			\paragraph{Motivation}
			I would like to do this project because after experiencing the 'rocking the boat' phase of COS 301 I find I have some ideas on how to better it. I think there are many areas which can be improved upon in order to ensure that 'rocking the boat' achieves all that it sets out to do.
		
		\newpage
		\subsection{Armand Maree}
			\begin{wrapfigure}{l}{5.1cm}
				\begin{center}
					\includegraphics[width=5cm]{armand.jpg}
				\end{center}
			\end{wrapfigure}
			\paragraph{Interests and Hobbies}
			During my off time I like to socialize with friends and enjoy watching sports. I also like solving puzzles to keep my brain active during holidays.\\
			Tutoring scholars and university students has become a passion for me. I always look forward to these sessions.
			
			\paragraph{Technical Skills}
			I am good at solving complex problems and building data structures. I believe this is a valuable skill to complete any project, especially in the field of computer science.
			
			\paragraph{Past Experience}
			I have developed websites for other start up companies and I also have a website of my own (\href{http://www.codehaven.co.za}{www.codehaven.co.za}).
			I also have some Android developing experience I gained from side projects.
			
			\paragraph{Non-Technical Strengths}
			\begin{itemize}
				\setlength\itemsep{0.2em}
				\item Good leader
				\item Fast learner
				\item Team player
				\item Good communicator
				\item Passionate
				\item Problem solver
			\end{itemize}
			
			\paragraph{Motivation}
			THIS SECTION IS PROJECT SPECIFIC
			
	\newpage
	\section{Project Execution}
		\subsection{Development Methodology}
			\paragraph\indent
			We are planning on using the Agile iterative software development methodology. The reason we have chosen this methodology can be described through the benefits of this methodology:
			\begin{itemize}
				\setlength\itemsep{0.2em}
				\item High degree of collaboration between the client and project team
				\item Allows clients to be involved throughout the project - this requires clients to understand that the work they will see is a 'work in progress'
				\item By using the idea of Sprints new features are delivered quickly and frequently
				\item Focusing on users needs results in each feature incrementally delivering value not only an IT component
				\item The breaking down of the projects into units allows the team to focus of high-quality development, testing and collaboration. Quality is improved by finding and fixing bugs quickly, and realising expectation mismatched quickly
			\end{itemize}
			
			more information on the benefits of this methodology can be found at: \sloppy\url{http://www.seguetech.com/blog/2013/04/12/8-benefits-of-agile-software-development}
			This methodology will allow us to frequently display working progress of the desired system to the client. It will also allow us to have larger, but still manageable, portions of the work done between each meeting. We believe this is essential in order to make faster progress while still being able to make changes to the system should the requirements change. See figure \ref{fig:developmentMethodologies}.
			
			\begin{figure}[!h]
				\includegraphics[width=\linewidth]{developmentMethodologies.png}
				\caption{Waterfall vs Iterative vs Extreme Programming methodologies.}
				\label{fig:developmentMethodologies}
			\end{figure}
		
		\subsection{Client Updates}
			\paragraph\indent
			 As the client of this project is Dr Linda Marshall who is employed by the university frequent face-to-face meetings could be arranged in order for Dr Marshall and developers (students) to discuss important milestones in the project, should it be necessary. Regularly updates (weekly or fortnightly) can be made known to the client via email. We could make use of a tasking system in which we set a number of tasks we wish to achieve and make this available to the client in order for them to monitor our progress.
			
		\subsection{Initial Ideas}
			\paragraph\indent
			We intend to begin this project by gathering requirements from all stakeholders and doing a thorough analysis of these requirements. We will also determine any 'nice-to-have' aspects stakeholders may want in the system. We will then create an Analysis an Design specification which will provide the client with information regarding
			\begin{itemize}
				\item Access channel requirements
				\item Quality requirements
				\item Integration requirements
				\item Architecture constraints
				\item Use Case prioritization
				\item Required functionality depicted by use cases
				\item Process specification (more detailed steps of a use case)
				\item Domain Model
				\item Open issues we may have discovered
			\end{itemize}
			this document will be provided to Dr Marshall to ensure that our plans for the software encorporate all the aspects they want in the system. Any feedback from Dr Marshall regarding the above mentioned aspects will be incorporated into our design specification. We will also supply Dr Marshall with a document regarding software architecture specifications. Any feedback on this will also be incorporated in the specification.\\

			We will then create a list of milestones to be achieved throughout this project with deadlines by which we hope to achieve these milestones. Some initial ideas regarding these milestones may be:
			\begin{itemize}
				\item Creating an algorithm
				\item Creating a Design for the Graphical User Interface
				\item Creating a Web interface
				\item ....Add items here...
			\end{itemize}
		
		\subsection{Potential Technologies}
			\paragraph\indent
			\begin{itemize}
    		        \item   Git and GitHub: A distributive version control system that is easy to use and free. It will be used to store and control the code written for the project and thus all code written by group members will be easily managed. (http://github.com/). 
    		        \item PostgreSQL: Since we need to have a data store, we could use a PostgreSQL database to store the data.It is a free, open-source, cross-platform, object-orientated database management system. (http://www.postgresql.org/about/). An added benefit is the fact that it has an unlimited database size unlike many other similar technologies.
    		        \item HTTPS: This technology is almost a must as it will increase the systems security by adding needed encryption.
    		        \item Bootstrap: This technology is a powerful mobile first front-end framework for faster and easier web development. It will standardize the way content is displayed for the web front end.
    		        \item SMTP: An extra for the requirements, but something that could be integral, is the protocol needed to send emails such as notifications and reminders.
    		        \item JavaScript/JQuery: Used for client side functionality for example verification of user data and passwords
    		        \item TomEE Application Server: The Apache Tomcat software is an open source implementation of the Java Servlet, JavaServer Pages, Java Expression Language and Java WebSocket technologies. This will provide us with the environment in which the server will run. (\url{http://tomcat.apache.org/})
    		        \item  JUnit Testing: A unit testing framework for the Java programming language.
    		    \end{itemize}
		
		\subsection{Deliverables}
			\paragraph\indent
			 On completion of this project, the following deliverables will be presented to the client
		    \begin{itemize}
		        \item a web interface
		        \item all code
		        \item detailed documentation of the code and how it works
		        \item a user manual on how to set up and use the system in an efficient and effective manner
		    \end{itemize}
\end{document}
