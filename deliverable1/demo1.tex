\documentclass[hidelinks,english]{article}

\usepackage{graphicx}
\usepackage{grffile}
\usepackage[T1]{fontenc}
\usepackage{babel}
\usepackage{wrapfig}
\usepackage{hyperref}

\date{\today}

\graphicspath{{Pictures/}}
\begin{document}	
	\begin{titlepage}
		\pagenumbering{gobble}
		\begin{figure}[!t]
			\includegraphics[width=\linewidth]{up_logo.png}
		\end{figure}
		\vspace*{\stretch{1.0}}
		\begin{center}
			\huge{Software Requirements Specification and Technology Neutral Process Design\\}
			\huge{Mindmap PIM}\\
			\large{Client: IMINISYS}\\
			\vspace{10mm}
			\huge{Team: A-Cube-N}\\
		\end{center}
		\begin{center}
			\begin{tabular}{ c c c }
				Dunkley, Nathan & Grobler, Arno & Lochner, Amy \\
				\texttt{14145759} & \texttt{14011396} & \texttt{14038600}\\
				& Maree, Armand &\\
				& \texttt{12017800} &
			\end{tabular}
		\end{center}
		\begin{center}
			Department of Computer Science, University of Pretoria
		\end{center}
		\vspace*{\stretch{2.0}}
	\end{titlepage}
	\newpage
	\tableofcontents
	\newpage
	\pagenumbering{arabic}
	
	\section{Introduction}
	This project was project was proposed to the University by a company with the name of IMINSYS. The primary supervisor of this project is Morkel Theunissen.
	
	
	\section{Vision}
	The vision of this project is to create an application that extracts data from various existing systems such as Facebook, Gmail etc; the application will then construct a mind map of this data. The user will be able to navigate and expand the nodes known as "bubbles". The main nodes will expand to show posts, images, emails etc that you have marked as important, were tagged in etc. This information will be displayed in an interactive website. The mind map will have filtering capabilities and basic transformations which can be applied to the mind map.
	
	\section{Background}
		\subsection{The client's problem}
		
		
		\subsection{Future business/research opportunities}
	
	
	\section{Architecture Requirements}
	
		We plan on using the following architectures
		\begin{itemize}
		\item Tom-EE
		\item REST 
		\item MVC
		\end{itemize}
		
		\subsection{Access Channel Requirements}
		
		
		\subsection{Quality Requirements}
		
		
		\subsection{Integration Requirements}
		We will be integrating our application with
		\begin{itemize}
		\item Facebook
		\item LinkedIn
		\item Gmail
		\item Google Calendar
		\item Google Notes
		\item Hangouts
		\end{itemize}
		
		\subsection{Architecture Constraints}
	
	
	\section{Functional Requirements and Application Design}
	
		\begin{itemize}
		\item Login System
		\item Choose Data Sources
		\item Set Branching Factor
		\item Share Node
		\item View Facebook Likes
		\item Comment on Facebook
		\end{itemize}
		
		
		\subsection{Use Case/Services Contracts}
	
	
		\subsection{Required Functionality}
	
	
	\section{Open Issues}
	
\end{document}
