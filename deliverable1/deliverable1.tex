\documentclass[hidelinks,english]{article}

\usepackage{graphicx}
\usepackage{grffile}
\usepackage[T1]{fontenc}
\usepackage{babel}
\usepackage{wrapfig}
\usepackage{hyperref}

\date{\today}

\graphicspath{{Pictures/}}
\begin{document}	
	\begin{titlepage}
		\pagenumbering{gobble}
		\begin{figure}[!t]
			\includegraphics[width=\linewidth]{up_logo.png}
		\end{figure}
		\vspace*{\stretch{1.0}}
		\begin{center}
			\huge{Software Requirements Specification and Technology Neutral Process Design\\}
			\huge{Mindmap PIM}\\
			\large{Client: IMINISYS}\\
			\vspace{10mm}
			\huge{Team: A-Cube-N}\\
		\end{center}
		\begin{center}
			\begin{tabular}{ c c c }
				Dunkley, Nathan & Grobler, Arno & Lochner, Amy \\
				\texttt{14145759} & \texttt{14011396} & \texttt{14038600}\\
				& Maree, Armand &\\
				& \texttt{12017800} &
			\end{tabular}
		\end{center}
		\begin{center}
			Department of Computer Science, University of Pretoria
		\end{center}
		\vspace*{\stretch{2.0}}
	\end{titlepage}
	\newpage
	\tableofcontents
	\newpage
	\pagenumbering{arabic}
	
	\section{Introduction}
	This project was project was proposed to the University by a company with the name of IMINSYS. The primary supervisor of this project is Morkel Theunissen.
	
	
	\section{Vision}
	The vision of this project is to create an application that extracts data from various existing systems such as Facebook, Gmail etc; the application will then construct a mind map of this data. The user will be able to navigate and expand the nodes known as "bubbles". The main nodes will expand to show posts, images, emails etc that you have marked as important, were tagged in etc. This information will be displayed in an interactive website. The mind map will have filtering capabilities and basic transformations which can be applied to the mind map.
	
	\section{Background}
		\subsection{The client's problem}
		
		
		\subsection{Future business/research opportunities}
		
		
	\section{Scope}
	
	
	\section{Software Architecture Overview}
	
	
	\section{Architecture Requirements}
		
		\subsection{Access Channel Requirements}
		
		
		\subsection{Quality Requirements}
		
		
		\subsection{Integration Requirements}
		We will be integrating our application with
		\begin{itemize}
			\item Facebook
			\item LinkedIn
			\item Gmail
			\item Google Calendar
			\item Google Notes
			\item Hangouts
		\end{itemize}
		
		\subsection{Architecture Constraints}
			Technical constraints include:
			\begin{itemize}
				\item \textbf{Programming language} We will be developing the system in Java.
				\item \textbf{Operating system} The server will be run off a Linux machine. 
			\end{itemize}
			Further there are no constraints and the architects have free roam to implement an architecture that suits their needs.
			
	\section{Architecture Design}
	
		\subsection{Architectural components addressing architectural responsibilities}
		
		\subsection{Infrastructure}
		
		\subsection{Tactics}
		
	\section{Database and Persistence}
	
	\section{Functional Requirements}		
		
		\subsection{Use Case/Services Contracts}
	
	
		\subsection{Required Functionality}
			Functionality that should be included in the system includes:
			\begin{itemize}
			\item A login/sign up system. 
			\item Adding various PIMs to your account.
			\item Interaction with PIMs directly from the mind map itself (commenting, emailing, etc).
			\item Expanding bubbles in order to find articles related to that bubble.
			\item Rapidly integrate new information into the system.
			\end{itemize}
	
	\section{Open Issues}
	
\end{document}
