\documentclass{beamer}

\mode<presentation> {
	\usetheme{Rochester}
	\setbeamertemplate{footline}[page number]
}

\usepackage{graphicx}
\usepackage{booktabs}

\title[Mind Map PIM]{Mind Map PIM}

\author{A-Cube-N}
\institute[UP]{
	Department of Computer Science, University of Pretoria
}
\date{\today}
\graphicspath{{pictures/}}

\begin{document}

\begin{frame}
	\titlepage
\end{frame}

\begin{frame}
	\frametitle{Overview}
	\tableofcontents
\end{frame}

\section{Concept}
	\subsection{Overview}
		\begin{frame}
		\frametitle{Project Overview}
		Mind Mapped PIM is a program that will process existing PIM sources such as Facebook and Gmail and construct a mind map using that data. 
		The mind map will allow functionality of the sources e.g. send an email. It will be programmed in a way to display information that is most relevant to you at a particular time
		\end{frame}
	
	\subsection{What it is not}
		\begin{frame}
		\frametitle{What it is not}
		The Mind Map PIM is not a new social media platform. Rather it acts as a filter for current your current social media platforms.  It will use intelligent algorithms to determine what information you would most likely want to see.
		\end{frame}
		
\section{Software Architecture}
	\subsection{Overview}
		\begin{frame}
		\frametitle{Software Architecture Overview}
			We will be using a layered system architecture to implement the system. This will allow us to better achieve:
			\begin{itemize}
				\item modularity
				\item flexibility
				\item scalability
				\item access limitation
			\end{itemize}
		\end{frame}
		
	\subsection{Requirements}
		\begin{frame}
		\frametitle{Software Architecture Requirements}
		
		\end{frame}
		
	\subsection{Design}
		\begin{frame}
		\frametitle{Software Architecture Design}
		
		\end{frame}
		
	\subsection{Persistence}
		\begin{frame}
		\frametitle{Software Architecture Persistence}
			For persistence we will be using MongoDB. This choice was made on the fact that it will integrate better with our OO system and also the fact that services like Mind Map PIM has the potential to attract many users.
			Benefits include:
			\begin{itemize}
				\item high write loads
				\item good for Big Data scenarios
				\item highly scalable
				\item document orientated storage
			\end{itemize}
		\end{frame}
		
	\subsection{Interface}
		\begin{frame}
		\frametitle{Software Architecture Interface}
			The system will be accessed via a web interface (our primary goal) and an Android app (as a secondary goal). The mind map will be presented in a fashion similar to the way that musicroamer.com displays related music artists (see next slide).
			
			\begin{itemize}
				\item At the center is the root node.
				\item Each social media platform has a node going out from the root node.
				\item Relevant information is displayed about each platform.
				\item User can expand any node to find relevant information about that node.
				\item The user can specify a ply depth to specify how many 'branches' the mind map should allow from a main node
				\item Should a node allow functionality e.g. Facebook node allows you to comment, this functionality will be provided in a panel to the right of the mind map
			\end{itemize}
		\end{frame}
		\begin{frame}
		\frametitle{Software Architecture Interface}
			\begin{figure}
				\includegraphics[scale=0.3]{musicroamer.png}
				\caption{Music Roamer layout of a music mind map}
			\end{figure}
		\end{frame}

\begin{frame}
	\Huge{\centerline{The End}}
\end{frame}

\end{document} 